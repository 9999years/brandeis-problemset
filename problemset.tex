\documentclass{ltxdoc}
\usepackage{hypdoc}
\usepackage[
	loadHyperref=true,
	createIndexEntries=false
]{doctools}
\usepackage{xparse}
\usepackage{hologo}

\newcommand{\email}[1]{\href{mailto:#1}{\texttt{#1}}}
\newcommand{\ps}{\ltxclass{problemset}}
\newcommand{\cosi}[1]{\textsc{cosi}~#1}
\newcommand{\todo}[1]{\begin{itemize}\item[\textbf{To-do:}] #1 \end{itemize}}
\newcommand{\note}[1]{\begin{itemize}\item[\textbf{Note:}] #1 \end{itemize}}
\newcommand{\ctan}[1]{\href{https://ctan.org/pkg/#1}{\package{#1}}}
\NewExpandableDocumentCommand{\Th}{O{l} m}
	{\multicolumn{1}{#1}{\textbf{#2}}}

\author{Rebecca Turner\thanks{Brandeis University; \email{rebeccaturner@brandeis.edu}}}
\title{The \ps\ Document Class}
\date{2018-10-18}

\begin{document}
\maketitle

\begin{abstract}

	Brandeis University's computer science (``\textsc{cosi}'') courses
	often assign ``problem sets'' which require fairly rigorous
	formatting. The \ps\ document class, which extends
	\ltxclass{article}, provides a simple way to typeset these problem
	sets in \LaTeX.

\end{abstract}

\tableofcontents
\vfill
\pagebreak

\section{Class configuration}

\subsection{Class options}

\ps\ defines a limited set of key-value options that may be set at
\cs{documentclass}-time. These may be removed entirely in a future release,
as it seems ``messy'' to have three configuration methods
(\cs{documentclass} options, \cs{problemsetsetup}, and singular option
commands).

\begin{Optionlist}
	duedate & Assignment due date in full \\
	instructor & Instructor name in full \\
	course & Course name in full \\
	assignment & Assignment name in full \\
\end{Optionlist}

Given that \cs{documentclass} option parsing is much more limited than other
key-value interfaces, these options have limited capabilities.

\subsection{Setting options after loading \ps}

\begin{macro}{\problemsetsetup}\marg{options} sets global \ps\ options; see
table~\ref{tab:problemset-options} for a list of valid options.

\begin{table}[h]
	\centering
	\caption{Options for \cs{problemsetsetup}; many of these are just used
	in document headers.}
	\label{tab:problemset-options}
	\begin{Optionlist}
	course & Course name in full. \\
	coursenumber & Course name shorthand; use \texttt{21a} for
		``\cosi{21a}''. \\
	assignment & Assignment name in full. \\
	number & Assignment name shorthand; use \texttt{3} for ``Problem Set
		3''. \\
	duedate & Due date, e.g.\ \texttt{2018-10-18}; not parsed at all. \\
	instructor & Course instructor. \\
	\end{Optionlist}
\end{table}
\end{macro}

\ps\ additionally provides a number of configuration commands with similar
interfaces as the \TeX\ macros \cs{author}, \cs{title}, and \cs{date}. It is
likely that these will be deprecated in the future.

\begin{macro}{\duedate}\marg{date} sets the due date in full.\end{macro}
\begin{macro}{\instructor}\marg{name} sets the instructor name.\end{macro}
\begin{macro}{\course}\marg{name} sets the course name in full.\end{macro}
\begin{macro}{\coursenumber}\marg{number} sets the course name by number; e.g.
	|\coursenumber{21a}| gives a course of ``\cosi{21a}''.\end{macro}
\begin{macro}{\assignment}\marg{name} sets the assignment name in full.\end{macro}
\begin{macro}{\problemsetnumber}\marg{number} sets the assignment name by
	number; e.g. |\problemsetnumber{3}| gives an assignment of ``Problem
	Set 3''.\end{macro}

\section{User commands and environments}

\ps\ provides a number of commands for typesetting problems.

\begin{macro}{problem}\oarg{options} defines a problem. A problem is set
1~inch from the left margin (although this amount may be customized by
modifying the \cs{problemindent} length) and begins a new page.

\begin{Optionlist}
title & A problem title, to be displayed after ``Problem'' and the problem's
	number.\\
number & A problem number; if given, the problem-number counter
	will not advance. The number must be robust, because it goes inside
	a \cs{section}. \\
pagebreak & True/false (default: true). Add a pagebreak before the problem? \\
\end{Optionlist}

Vertical material is allowed in a~\env{problem}.
\end{macro}

\begin{macro}{\subproblem}\oarg{description} prints a sub-problem, i.e.\ a
\cs{subsection}. It doesn't do very much at the moment.
\end{macro}

\begin{macro}{\Th}\oarg{colspec}\marg{header} prints a table-header in bold.
By default, the header is left-aligned, but arbitrary alignments can be
specified with \meta{colspec}. \cs{Th} is backed by \cs{multicolumn}.
\end{macro}

\begin{macro}{pseudocode}\oarg{keywords} prints
pseudocode.\footnote{Designed for \cosi{21a} as taught by Dr.\ Antonella
DiLillo} Several shortcuts are defined, as shown in
table~\ref{tab:pseudocode}. These shortcuts display in
\cs{pseudocodesymbolfont} (default: \cs{ttfamily}), which may be redefined
if you prefer something else. I find the \ctan{MnSymbol} package has symbols
which look good with Courier, but difficulties loading the actual font made
me resort to loading it via the \ctan{fontspec} package.

\begin{table}[h]
	\centering
	\caption{Shortcuts provided by the \env{pseudocode} environment}
	\label{tab:pseudocode}
	\begin{tabular}{>{\ttfamily}lll}
	\Th{Input} & \Th{Display} & \Th{Codepoint} \\
	<-         & ←            & U+2190 \\
	->         & →            & U+2192 \\
	(/)        & $\emptyset$  & U+2205 \\
	inf        & $\infty$       & U+221E \\
	!=         & $\ne$        & U+2260 \\
	>=         & $\ge$        & U+2265 \\
	<=         & $\le$        & U+2264 \\
	\end{tabular}
\end{table}

\todo{Improve the font selection mechanism; maybe provide a command for each
	symbol?}
\note{If your \TeX\ engine doesn't support \textsc{utf}-8 input, the
	shortcuts might appear totally blank or garbled. Good luck! It will surely
	work with \hologo{XeTeX} and probably work with \hologo{LuaTeX}.}

\begin{latexcode}
% the optional [Bar] makes [Bar] bold like the other keywords
\begin{pseudocode}[Bar]
Bar(a, n)
    Input:  two integers, a and n
    Output: a^n
    k <- n # k is a counter
    b <- 1
    c <- a
    while k > 0 do
        if k mod 2 = 0 then
            k <- k/2
            c <- c * c
        else
            k <- k - 1
            b <- b * c
    return b
\end{pseudocode}
\end{latexcode}
\end{macro}

\begin{macro}{assembly}\oarg{extra options} typesets assembly
code.\footnote{Designed for \cosi{131a} as taught by Dr.\ Liuba Shrira}
Several considerations are taken into account; most notably, line numbers
are printed as \texttt{x + n}, where $n$ starts at 0 and counts by 4; the
line number actually indicates the instruction's location in memory as an
offset from the program start. Additionally, all valid instructions are
treated as keywords and styled appropriately.

Any extra options are passed directly to the \ctan{listings} package.

\begin{latexcode}
\begin{assembly}
        LOAD  R4, $200       ; sum addr
        LOAD  R1, =0         ; sum
        LOAD  R2, =0         ; i
        LOAD  R3, =0         ; j
        BR    OUTER          ; we know i < 10
INNER:  ADD   R1, R3         ; sum += j
        INC   R3             ; j++
OUTER:  BLT   R3, R2, INNER  ; while j < i goto inner
        INC   R2             ; i++
        LOAD  R3, =0         ; j = 0
        BLT   R2, =10, OUTER ; while i < 10
        STORE R1, @R4        ; store sum into sum address
        HALT
\end{assembly}
\end{latexcode}
\end{macro}

\begin{macro}{java}\oarg{extra options} Tragically-common shorthand
environment for a listing of Java code.

Any extra options are passed directly to the \ctan{listings} package.

\end{macro}

\section{Future work}

It might be nice to allow passing options to \ps\ to match certain teachers'
preferred styles, e.g.\ \option{antonella} or \option{liuba}.

Improving font selection; what should the default fonts be? Should
\hologo{XeTeX} be mandatory?

\end{document}
