\PassOptionsToPackage{inline}{enumitem}
\documentclass{ltxguidex}
\usepackage{fontspec}
\usepackage[pseudocode, solution, assembly, tabu]{brandeis-problemset}

\usepackage{xkeyval}
\usepackage{xparse}
\usepackage{textcomp} % needed for listings
\usepackage{hologo} % xetex, etc. logos
\usepackage{changelog}
\newcommand{\bps}{\ltxclass{brandeis-problemset}}
\newcommand{\fontspecok}{\hologo{XeLaTeX} or \hologo{LuaTeX}}
\newcommand{\unicode}[1]{\textsc{u+#1}}
\NewDocumentCommand{\cosi}{o m}{\textsc{cosi}~#2\IfValueT{#1}{ (#1)}}
\newnotice{todo}{To-do}

\newcommand{\reqopt}[1]{Requires the \texttt{\color{magenta}#1} option.\par}
\newcommand{\dep}[1]{{\noticestyle Warning:} deprecated since #1.\par}
\newcommand{\withopt}[1]{\fbox{With the \option{#1} option:}}

\makeatletter
\renewcommand{\SX@explpreset}{
	language=[LaTeX]TeX,
	numbers=none,
}
\define@cmdkeys{joinlist}{sep, wrapper}
\define@cmdkey{joinlist}{last}{\KV@joinlist@oxfordfalse}
\define@boolkey{joinlist}{oxford}[true]{}
\define@boolkey{joinlist}{and}[true]{\ifKV@joinlist@and\def\cmdKV@joinlist@last{and}\fi}
\define@boolkey{joinlist}{or} [true]{\ifKV@joinlist@or \def\cmdKV@joinlist@last{or}\fi}
\presetkeys{joinlist}{sep={,}, last=and, oxford, wrapper=\relax}{}
\NewDocumentCommand{\joinlist}{O{} >{\SplitList{,}} m}
	{{\setkeys{joinlist}{#1}%
	\newcommand{\joinlist@wrapper}[1]{\item\cmdKV@joinlist@wrapper{##1}}%
	\begin{itemize*}[
		label={},
		itemjoin={\cmdKV@joinlist@sep},
		itemjoin*={\ifKV@joinlist@oxford\cmdKV@joinlist@sep\ \fi
			\cmdKV@joinlist@last}]%
		\ProcessList{#2}{\joinlist@wrapper}%
	\end{itemize*}}}
\makeatother

\makeatletter
\NewDocumentCommand{\withopts}{m}
	{\fbox{With any of the \joinlist[or, wrapper=\option]{#1} options:}}
\makeatother
\newcommand{\onlyclass}{Only available in the \bps\ document class.}
\newcommand{\optenables}[2]{If \option{#1} is set, then \option{#2} is
automatically enabled.}

\setmainfont{Tiempos Text}
\usepackage{FiraSans}
\usepackage{FiraMono}
\lstset{basewidth=0.6em}

\lstnewenvironment{latexfile}[1]
	{\lstset{
		language=[LaTeX]TeX,
		numbers=none,
	}}
	{}

\author{Rebecca Turner\thanks{Brandeis University; \email{rebeccaturner@brandeis.edu}}}
\title{The \bps\ Document Class}
\date{2019/03/11 0.5.2}
\begin{document}
\maketitle
\begin{abstract}

	Brandeis University's computer science (``\textsc{cosi}'') courses
	often assign ``problem sets'' which require fairly rigorous
	formatting. The \ctan{brandeis-problemset} document class, which
	extends \ltxclass{article}, provides a simple way to typeset these
	problem sets in \LaTeX.

	Although \bps\ is compatible with all \LaTeX\ flavors, \fontspecok\
	is recommended for \ctan{fontspec} support.

\end{abstract}

\begin{note}
	The \bps\ document class should be considered experimental; the only
	stable \textsc{api} is that of the |problem| environment.
\end{note}

\begin{note}
	Browse the sources, contribute, or complain at \\
	\https{github.com/9999years/brandeis-problemset}
\end{note}

\begin{note}
	In roughly May 2021 I will graduate from Brandeis University and
	this package will become unmaintained. Although I'd like the
	computer science department to adopt this package, I'm not sure any
	professors have an interest in that maintenance. If you care about
	the extended future of this package, send some emails!
\end{note}

\tableofcontents
\vfill
\pagebreak

\section{Default behavior}
\bps\ provides packages and well-formatted constructs (notably the
|problem| environment) for problem-set writers. \bps\ will always render
its body copy as a Times variant (\ctan{stix} for plain \LaTeX\ or
\ctan{xits} with \fontspecok) and always contains a useful header (which
contains the page number, author's name, course, instructor, and assignment
due date).

\subsection{Default packages loaded}
In general, \bps\ will only load a feature if it's useful in more than one
class; features like Gantt charts, Scheme code, assembly code, and so on,
are loaded on demand.

\begin{enumerate}
\item \ctan{hyperref}, for a nicely-linked table of contents;
	|\href||{url}{label}|.
\item \ctan{geometry} for page size and margins.
\item \ctan{enumitem} for better control over the margins and spacing of
	the |enumerate|, |itemize|, and |description|
	environments.
\item \withopts{listings, scheme, pseudocode, assembly}
\begin{enumerate}
	\item \ctan{listings}, for verbatim code listings (including the
		|assembly|, |java|, and |pseudocode| environments).
	\item \ctan{xcolor}, for colored identifiers, strings, comments, and
		line numbers in listings; e.g.\ |\color||{gray}|.
\end{enumerate}
\item \withopt{math} \begin{enumerate}
	\item \ctan{amsmath} for tons of useful math commands, including
		|\text|, |\intertext|, and |\boxed| as well as the |bmatrix|,
		|multiline|, |gather|, |align|, and
		|alignat| environments. See
		\href{http://texdoc.net/texmf-dist/doc/latex/amsmath/amsldoc.pdf}{``User's
		Guide for the \package{amsmath} Package''} for a more
		complete listing.
	\item \ctan{mathtools} for other useful/utilitarian commands.
	\item With \fontspecok, \ctan{unicode-math} to allow equations to be
		copy-pasted.
	\item The \ctan{stix2-otf} math fonts.
\end{enumerate}
\item \withopt{tabu} \begin{enumerate}
	\item \ctan{multirow} for cells spanning multiple rows.
	\item \ctan{booktabs} for beautiful-by-default tables and the
		|\cline| macro.
	\item \ctan{tabu}, the best table package with dynamically resizable
		columns, easy creation of new column types, and more.
\end{enumerate}
\end{enumerate}

\section{Class configuration}

\subsection{Class options}

Class options are limited to configuration options which require the loading
of fonts or other packages; ``string'' settings like the assignment's due
date are configured with the |\bpsset| command.

All class options can be negated with |no<option>|; e.g.\ \option{scheme}
can be disabled with \option{noscheme}. Later options override previous
ones.

\begin{keys}
	\key{antonella} \onlyclass\ Use Dr.\ Antonella DiLillio's preferred
		styles (Courier for code)
	\key{tabu} \onlyclass\ Loads useful packages and defines commands for
		typesetting tables.
	\key{math} \onlyclass\ Loads utilities for typesetting mathematics.
		Because mathematics requires specialized fonts\footnote{The
		\textsc{stix2} math font that this package loads weighs in
		at 740kb!} as well as several other specialized
		packages,\footnote{Including \ctan{amsmath},
		\ctan{mathtools}, and, with \fontspecok,
		\ctan{unicode-math}.} this is disabled by default. Although
		I find myself needing mathematics fonts frequently, I know
		many students do not.
	\key[default=true]{listings} Enables code listings and enables
		default styles for colored keywords, pretty line numbers,
		and so on. While only this option enables the |java|
		environment, the \option{scheme}, \option{assembly}, and
		\option{pseudocode} options also load and style the
		\ctan{listings} package.
	\key{scheme} Enable Scheme language support, in particular for the
		\textsc{r5rs} dialect. Additionally, provides the |scheme|
		environment.
	\key{pseudocode} Enables the |pseudocode| environment, notably
		useful for \cosi[data structures]{21b}.
	\key{assembly} Enables the |assembly| environment, notably useful
		for \cosi[operating systems]{131a}.
	\key{solution} Show the content of |solution| environments; by
		default, they are excluded from compilation using the
		\ctan{comment} package.
	\key{gantt} Loads the |ganttschedule| environment.
	\key[default=true]{maketitle} Redefines |\maketitle| and defines
		|\maketitlepage| to include information about the course,
		instructor, assignment, and due date.
		\optenables{maketitle}{option}.
	\key[default=true]{header} Adds a header to the top of every page including
		information about the author, assignment, due date, and
		instructor. \optenables{header}{option}.
	\key[default=true]{config} Enables document metadata like the
		assignment's due date, name/number, instructor, and course
		through the |\bpsset| command, which supports |\maketitle|,
		page headers, and so on.
\end{keys}

\subsubsection{Class options or keyval options?}
There's trade-offs to be made either way; class options greatly simplify the
code (we don't have to worry about loading packages and defining commands in
more than one place, for the most part), but are much less flexible and
difficult to interface with when the package loading is hidden. At the
moment, package/class options provide commands and keyval options define
document-configuration data (like names, due dates, and the like; see
section~\ref{s:config} for more information).

\subsection{Configuring \bps\ after loading}\label{s:config}
\begin{desc}
|\bpsset{<options>}|
\end{desc}
Sets global \bps\ options. Mnemonic: \textbf{B}randeis \textbf{p}roblem
\textbf{s}et \textbf{set}up (this should feel familiar to users of the
\ctan{listings} package).

\begin{note}
	|\bpsset| was renamed from |\problemsetsetup| in version 0.5.0. The
	|\problemsetsetup| command is deprecated and will be removed in a
	future release.
\end{note}

\begin{keys}
\key{course}[\m{course name}]
	Course name in full.
\key{coursenumber}[\m{course number}]
	Course name shorthand; use \texttt{21a} for ``\cosi{21a}''.
\key{assignment}[\m{assignment name}]
	Assignment name in full.
\key{number}[\m{problem set number}]
	Assignment name shorthand; use \texttt{3} for ``Problem Set 3''.
\key{duedate}[\m{due date}]
	Due date, e.g.\ \texttt{2018-10-18}; not parsed at all, but
	\href{https://en.wikipedia.org/wiki/ISO_8601}{\textsc{iso} 8601
	dates} are highly recommended.
\key{instructor}[\m{course instructor}]
	Course instructor. With the \option{antonella} class-option, this is
	automatically set to |Dr.~Antonella DiLillio|.
\key{author}[\m{your name}]
	Alternate interface for the |\author| command.
\key{date}[\m{document date}]
	Alternate interface for the |\date| command.
\key{codefont}[\m{fontspec font name}]
	With \fontspecok, pass the given font to |\setmonofont| and enable
	Unicode shortcuts for the |pseudocode| environment. (If you need to
	specify options to |\setmonofont|, use |\setcodefont|.)
\end{keys}

\begin{desc}
|\setcodefont[<fontspec options>]{<fontspec font name>}|
\end{desc}
Sets the monospaced font to \meta{fontspec font name} and uses it for
shortcuts in the |pseudocode| environment.

\subsection{Practical usage}

You may find it useful to define a customized document class for each
course. There's no reason to install these to some system-wide directory; it
makes sense for them to live in the same directory as the problem set source
files. For instance, \filename{cosi21a.cls} might read:

\begin{latexfile}{cosi21a.cls}
\LoadClass[antonella, pseudocode]{brandeis-problemset}

% pass all unknown options to brandeis-problemset
\DeclareOption*{\PassOptionsToClass
	{\CurrentOption}{brandeis-problemset}}
\ProcessOptions\relax

% set course/author data
\bpsset{
	coursenumber=21a,
	author=Rebecca Turner,
}

% get a prettier code font -- these can be pretty big so they're not
% loaded by default
\setcodefont[
  Extension = .otf,
  UprightFont = *-Regular,
  BoldFont = *-Bold,
]{FiraMono}
\end{latexfile}

and then \filename{ps1.tex} might read:

\begin{latexfile}{ps1.tex}
\documentclass{cosi21a}
% stuff specific to this assignment
\bpsset{
	number=1,
	duedate=2018-10-29,
}
\begin{document}
% etc.
\end{document}
\end{latexfile}

Note that you could use e.g.~|\documentclass[math]{cosi21a}| to add a
specific per-document option. See section~\ref{sec:example} for a more
complete example.

\section{User commands and environments}

\bps\ provides a number of commands for typesetting problems.

\begin{desc}
|\begin{problem}[<options>]...\end{problem}|
\end{desc}
Defines a problem. A problem is set 1~inch from the left margin (although
this amount may be customized by modifying the |\problemindent| length) and
begins a new page.

\begin{note}
	The |problem| and |subproblem| environments are typeset using
	|\section| and |\subsection| respectively, and as such can be
	customized with styling packages like \ctan{titlesec}; note,
	however, that they ignore and hide the usual section counters.
\end{note}

|<options>| may include:
\begin{keys}
\key{title}[\m{problem title}]
	Displayed after ``Problem'' and the problem's number.
\key{number}[\m{problem number}]
	If given, the problem-number counter will not advance. The number
	must be robust, because it goes inside a |\section|.
\key{pagebreak}[\bool][true]
	Add a pagebreak before the problem?
\key{label}[\m{problem label}]
	Adds a custom label to the problem with |\label| that can be used
	with |\ref|. I recommend prefixing your problem labels with |p:| as
	in |p:big-o-proofs|.
\key{toc}[\bool][true]
	Include this problem in the table of contents?
\key{part}[\m{part name}]
	Indicates that this problem starts a new ``part'' of the assignment;
	actually calls |\part| under the hood.
\key{partlabel}[\m{part label}]
	Adds a custom label to this part in the same fashion as the
	\option{label} key.
\end{keys}

Vertical material is allowed in a~|problem|.

\begin{desc}
|\begin{subproblem}[<options>]...\end{subproblem}|
\end{desc}
Defines a subproblem. A subproblem is set 1~inch from the left margin
(although this amount may be customized by modifying the |\subproblemindent|
length) and begins a new page. Valid |<options>| are identical to the
|problem| environment's, with the following exceptions:
\begin{keys}
\key[default=false]{pagebreak}
	False by default.
\key{part}
\key{partlabel}
	Not available in the |subproblem| environment.
\end{keys}

\begin{example}
	Note that although these examples are too short to display them,
	vertical material --- including listings --- is allowed in the
	|problem| and |subproblem| environments.
\end{example}
\begin{LTXexample}[preset={
	\presetkeys{problem}{toc=false}{}
	\presetkeys{subproblem}{toc=false}{}
	}]
\begin{problem}
  This is a problem in an assignment.
\end{problem}

Some solution here\dots

\begin{subproblem}
  This is a subproblem.
\end{subproblem}
\end{LTXexample}

\begin{desc}
|\begin{solution}...\end{solution}|
\end{desc}

Defines a solution for a problem; a solution prints in blue and is excluded
from the compiled document entirely unless the \option{solutions} package
option is given.

In this way, the same \extension{tex} file can serve as both a postable
assignment prompt and an answer key.

\begin{note}
	The style of solutions is customizable by redefining
	|\solutionstyle|; it's defined to |\color{blue}| by default.
\end{note}

\begin{example}
	The default solution style is shown below; note, however, that this
	document is compiled with the \option{solution} option passed to
	\bps; without it, the typeset solution is entirely blank.
\end{example}
\begin{LTXexample}
\begin{solution}
  Some solution here\dots
\end{solution}
\end{LTXexample}

\begin{desc}
|\Th[<column spec>]{<header text>}|
\end{desc}

Typesets a table header in bold-face. |<column spec>| defaults to |l|.
Useful for when a column is wrapped in a math environment; if you have a
column |>>{\ttfamily}l|, using |\Th| will not print the header in
|\ttfamily|.

\begin{LTXexample}
\begin{tabular}{ll}
\Th{Server} & \Th{\textsc{ip}} \\
juice & 1.1.1.1 \\
dogs & 2.2.2.2 \\
\end{tabular}
\end{LTXexample}

\begin{desc}
|\begin{pseudocode}[<keywords>]...\end{pseudocode}|
\end{desc}
Prints pseudocode.\footnote{Designed for \cosi{21a} as taught by Dr.\
Antonella DiLillo}

Several ``shortcuts,'' which replace a source-code sequence like
\texttt{->} with a symbol like \texttt{→}, are shown in
table~\ref{tab:pseudocode}.

These shortcuts display in |\pseudocodesymbolfont| (default:
|\ttfamily|), which may be redefined if you prefer something else. The
easiest way to change |\pseudocodesymbolfont| is with |\setcodefont|. If
you use the \option{antonella} option with \fontspecok, \bps\ will load
\ctan{lm-math} and display the symbols seen in table~\ref{tab:pseudocode},
which look significantly better with Courier than \ctan{stix}' symbols.

The following words are treated as keywords in pseudocode, and will be
bolded as appropriate: |Input|, |Output|, |Complexity|, |while|, |do|,
|return|, |for|, |to|, |if|, |then|, |else|, |True|, |False|, |None|, |and|,
|or|, |nil|, and |len|.

\begin{table}[h]
	\centering
	\caption{Shortcuts provided by the \texttt{pseudocode}
	environment}%
	\label{tab:pseudocode}
	\begin{tabular}{llll}
	\Th{Input}    & \Th{Command}            & \Th{Display}          & \Th{Codepoint} \\
	\ttfamily <-  & |\pseudocodeleftarrow|  & \pseudocodeleftarrow  & \unicode{2190} \\
	\ttfamily ->  & |\pseudocoderightarrow| & \pseudocoderightarrow & \unicode{2192} \\
	\ttfamily (/) & |\pseudocodeemptyset|   & \pseudocodeemptyset   & \unicode{2205} \\
	\ttfamily inf & |\pseudocodeinfty|      & \pseudocodeinfty      & \unicode{221e} \\
	\ttfamily !=  & |\pseudocodene|         & \pseudocodene         & \unicode{2260} \\
	\ttfamily >=  & |\pseudocodege|         & \pseudocodege         & \unicode{2265} \\
	\ttfamily <=  & |\pseudocodele|         & \pseudocodele         & \unicode{2264} \\
	\end{tabular}
\end{table}

\begin{example}
	Note how the option |[Bar]| argument makes |Bar| appear bold like
	any other keyword in the typeset listing.
\end{example}
\begin{LTXexample}[rframe=]
\begin{pseudocode}[Bar]
Bar(a, n)
    Input:  two integers, a and n
    Output: a^n
    k <- n  # k is a counter
    b <- inf
    c <- a
    while k >= 0 do
        if k mod 2 = 0 then
            k <- k / 2
            c <- c * c
        else
            k <- k - 1
            b <- b * c
    return b
\end{pseudocode}
\end{LTXexample}

\begin{desc}
|\begin{assembly}[<listings options>]...\end{assembly}|
\end{desc}
Typesets assembly code.\footnote{Designed for \cosi{131a} as taught by Dr.\
Liuba Shrira} Several considerations are taken into account; most notably,
line numbers are printed as \texttt{x + n}, where $n$ starts at 0 and counts
by 4; the line number actually indicates the instruction's location in
memory as an offset from the program start. Additionally, all valid
instructions are treated as keywords and styled appropriately.

|<listings options>| is passed directly to the \ctan{listings} package.

\begin{LTXexample}[rframe=, wide,
	explpreset={basicstyle=\small\ttfamily, basewidth=0.575em,},
	preset=\lstset{numbers=left, basicstyle=\small\ttfamily, basewidth=0.575em,},]
\begin{assembly}
LOAD  R4, $200       ; sum addr
LOAD  R1, =0         ; sum
LOAD  R2, =0         ; i
LOAD  R3, =0         ; j
BR    OUTER          ; we know i < 10
INNER:
ADD   R1, R3         ; sum += j
INC   R3             ; j++
OUTER:
BLT   R3, R2, INNER  ; while j < i goto inner
INC   R2             ; i++
LOAD  R3, =0         ; j = 0
BLT   R2, =10, OUTER ; while i < 10 goto outer
STORE R1, @R4        ; store sum
HALT
\end{assembly}
\end{LTXexample}

\begin{desc}
|\begin{java}[<listings options>]...\end{java}|
\end{desc}
Tragically-common shorthand environment for a listing of Java code.

|<listings options>| is passed directly to the \ctan{listings} package.

\begin{desc}
|\begin{scheme}[<listings options>]...\end{scheme}|
\end{desc}
Shorthand environment for a listing of Scheme code, useful for \cosi{121b}.
Requires the \option{scheme} package option to be loaded.

|<listings options>| is passed directly to the \ctan{listings} package.

\begin{desc}
|\begin{ganttschedule}[<total cell count>]...\end{ganttschedule}|
\end{desc}
An environment for drawing Gantt charts indicating process scheduling. The
mandatory argument indicates how small the grid should be; \texttt{19}
subdivides the line into 19 cells.

To use the |ganttschedule| environment, make sure to use the
\option{gantt} package option.

Within a |ganttschedule|, use the |\burst| command to indicate an
active process (i.e.\ a process burst).

\begin{note}
	The charts |ganttschedule| draws aren't actually really proper Gantt
	charts, which can indicate parallel activities; however, that's what
	Liuba calls them, so that's what they're called here.
\end{note}

\begin{desc}
|\pid{<pid>}{<burst length>}|
\end{desc}
Draw a burst for process |<pid>| of time length |<burst length>|.

\begin{note}
	The Gantt chart packages (notably \ctan{tikz}) don't play nicely
	with \ctan{ltxguidex}, the package this documentation is written in;
	for a typeset example, see section~\ref{sec:example}.
\end{note}

\begin{latexcode}
\begin{ganttschedule}{19}
	\pid{2}{1}
	\pid{4}{1}
	\pid{3}{2}
	\pid{5}{5}
	\pid{1}{10}
\end{ganttschedule}
\end{latexcode}

\begin{note}
	Because |ganttschedule| relies on \ctan{tikz}, \ctan{fp}, and
	\ctan{calc}, it can add significantly to document compile times. If
	you intend to use the |ganttschedule| environment, make sure to use
	the \option{gantt} class option or set \option{gantt} in
	|\bpsset|. If you fail to include the \option{gantt}
	option, you will see an error message:

\begin{latexcode}
! Package brandeis-problemset Error: ganttschedule enviornment not loaded in preamble.

See the brandeis-problemset package documentation for explanation.
Type  H <return>  for immediate help.
l.4 \burst
	{1}{1}
? H
Did you mean to use the 'gantt' option for the brandeis-problemset document class?

\end{latexcode}
\end{note}

\subsection{General formatting commands}

\begin{desc}
|\newacronym[<command>]{<acronym text>}|\\
|\newacronyms{<acronym list>}|
\end{desc}

Creates a new acronym. If |<command>| isn't given, the text of the macro
will be used instead; |\newacronym{cfg}| would define a command |\cfg| which
typesets as ``\ac{cfg}''. If the
resulting command already exists, it will be redefined. For |\newacronyms|, the
|<acronym list>| is a comma-delimited list of acronyms.

\begin{LTXexample}
\newacronym[\xyz]{cfg}
A \xyz\ describes a context-free language\dots

\newacronyms{sparc, cpu}
The \sparc\ had a unique \cpu\dots
\end{LTXexample}


\begin{desc}
|\ac{<acronym>}|
\end{desc}
Typesets an acronym. The |<acronym>| should be lowercase (e.g.\ |\ac{cpu}|
rather than |\ac{CPU}|). Currently, |\ac| simply delegates to |\textsc|. In
the future, I'd like to support a bit of letterspacing; ``for abbreviations
and acronyms in the midst of normal text, use spaced small
caps.''\footnote{\textit{The Elements of Typographic Style} by Robert
Bringhurst, 2nd.\ ed, \S\ 3.2.2}

\begin{desc}|\Sc{<text>}|\end{desc}
An abbreviation for |\textsc|.
\begin{desc}|\Rm{<text>}|\end{desc}
An abbreviation for |\textrm|.
\begin{desc}|\Up{<text>}|\end{desc}
An abbreviation for |\textup|.
\begin{desc}|\Bf{<text>}|\end{desc}
An abbreviation for |\textbf|.
\begin{desc}|\It{<text>}|\end{desc}
An abbreviation for |\textit|.
\begin{desc}|\Tt{<text>}|\end{desc}
An abbreviation for |\texttt|.

\section{A class and a package}
As a user, you'll likely only need the \bps\ document class. However,
a \emph{package} named \bps\ is also provided. The class styles an entire
document, while the package only provides commands. This allows --- for
example --- loading the package for the examples in this document without
messing up our titles, headers, and so on.

\section{Example}\label{sec:example}

A brief example usage of \bps\ follows. For a longer, more in-depth example,
see
\href{https://github.com/9999years/brandeis-problemset/blob/master/example.tex}{\filename{example.tex}
in the \bps\ repository.}

\begin{latexcode}
\documentclass[gantt]{brandeis-problemset}
\author{Rebecca Turner}
\bpsset{
	coursenumber=21a,
	instructor=Dr.\ Liuba Shrira,
	duedate=2018-10-20,
	number=3,
}
\newcommand{\io}{\ac{io}}
\newcommand{\cpu}{\ac{cpu}}
\begin{document}

\begin{problem}
	Write an assembly program!
\end{problem}

\begin{assembly}
        LOAD  R1, $200      ; A = (program location) + 200
        LOAD  R2, =1        ; i = 1
\end{assembly}

\begin{problem}
	What does this algorithm do? Analyze its worst-case running time and
	express it using big-O notation.

\begin{pseudocode}[Foo]
Foo(a, n)
	Input:  two integers, a and n
	Output: a^n
	k <- 0
	b <- 1
	while k < n do
		k <- k + 1
		b <- b * a
	return b
\end{pseudocode}
\end{problem}

$\Rm{Foo}(a, n)$ computes $a^n$, and will run in $O(n)$ time always.

\begin{problem}[number=5.4]
	Consider the following set of processes, with the length of the
	\cpu\ burst given in milliseconds:

	\begin{center}
		\begin{tabu} to 0.25\linewidth{X[1,$]rr}
			\Th{Process} & \Th{Burst time} & \Th{Priority} \\
			P_1 & 10 & 3 \\
			P_2 & 1 & 1 \\
			P_3 & 2 & 3 \\
			P_4 & 1 & 4 \\
			P_5 & 5 & 2 \\
		\end{tabu}
	\end{center}%$

	Draw a Gantt chart to illustrate the execution of these processes
	using the \ac{sjf} scheduling algorith.
\end{problem}

\begin{ganttschedule}{19}
	\pid{2}{1}
	\pid{4}{1}
	\pid{3}{2}
	\pid{5}{5}
	\pid{1}{10}
\end{ganttschedule}
\end{document}
\end{latexcode}

\begin{changelog}[author=Rebecca Turner]
\begin{version}[v=Coming soon\dots]
\added
	\item Support for \cosi{130b}, including regular expressions and
	context-free grammars.
\end{version}

\shortversion{v=0.5.1, date=2019-03-09, changes={Distribution erroneously
	excluded \filename{brandeis-problemset.sty}.}}

\begin{version}[v=0.5.0, date=2019-03-06]
\added
	\item |\newacronym| and |\newacronyms| commands.
	\item Added \option{toc} option to the |problem| and |subproblem|
	environments.
\changed
	\item Cleaned up internals; improved option system, split \bps\ into
	a class and a package, renamed commands and lengths to use the
	|bps@| prefix more consistently.
	\item \bps\ now has a modular and much more conservative approach;
	far fewer packages are loaded and features are loaded only upon
	request to a much greater extent.
	\item |subproblem| is now an environment, not a command, with an
	interface matching the |problem| environment.
	\item The \option{assembly} and \option{pseudocode} options now
	define languages for the \ctan{listings} package, rather than just
	providing environments to use the languages.
\removed
	\item The following commands have been removed in favor of the
		|\bpsset| command, which encompasses their functionality
		entirely:
		|\duedate{<date>}|,
		|\instructor{<name>}|,
		|\course{<name>}|,
		|\coursenumber{<number>}|,
		|\assignment{<name>}|, and
		|\problemsetnumber{<number>}|.
	\item Removed dependencies: \ctan{environ} and \ctan{titlesec}. No
		longer unconditionally loaded: \ctan{hyperref},
		\ctan{xcolor}, \ctan{comment}, \ctan{listings},
		\ctan{multirow}, \ctan{booktabs}, \ctan{longtable},
		\ctan{tabu}.
\deprecated
	\item The |\problemsetsetup| command, which has been renamed
		|\bpsset|.
\end{version}

\begin{version}[v=0.4.4, date=2019-02-14]
\changed
	\item Changed Times body copy font from \ctan{tex-gyre}'s Termes to
	the newer \ctan{stix2-otf} (for \fontspecok) and \ctan{stix2-type1}
	(for other \TeX\ engines) --- the \textsc{stix2} fonts are somewhat
	unique amongst Times-likes in that they contain small caps.

	\item Redefined |\Re| to print in blackboard-bold.
\end{version}

\shortversion{v=0.4.3, date=2019-01-20,
	changes={Fixed typos in license file, fixed distributed
	documentation \extension{pdf}.}}

\begin{version}[v=0.4.2, date=2019-01-19]
\added
	\item |author| and |date| keys added to |\problemsetsetup| to
	simplify class-wide configuration.
\fixed
	\item Fixed definitions for |\duedate|, |\instructor|, etc.\ to
	avoid spurious errors due to undefined commands.
\changed
	\item Translated documentation to the new \ctan{ltxguidex} document
	class for added beauty.
	\item Re-licensed \bps\ to the \textsc{lppl} v1.3c for easy transfer of
	maintenence in the future.
\end{version}

\shortversion{v=0.4.1, date=2019-01-03,
	changes={Updated |scheme| environment to properly recognize all
	primitive functions, added syntax coloring to all code.}}

\begin{version}[v=0.4.0, date=2018-12-20]
\added
	\item |solution| environment and \option{solutions} class
		option.
	\item |scheme| shorthand environment and \option{scheme} class
		option.
\fixed
	\item Boolean class options being overwritten by keys defined for
		|\problemsetsetup|.
	\item Title-formatting errors
\removed
	\item Assignment- and course-specific class options
		\option{duedate}, \option{assignment}, \option{instructor},
		and \option{course}. These settings should be configured
		with either |\problemsetsetup| or their specific commands.
		(|\duedate|, |\instructor|, etc.).
\end{version}

\begin{version}[v=0.3.0, date=2018-10-24]
\added
	\item This changelog.
	\item Support for |\part|s and referencing problems.
	\item Options to |problem| environment: \option{part},
		\option{label}, and \option{partlabel}.
	\item |\maketitle| (contrast with |\maketitlepage|).
\end{version}

\begin{version}[v=0.2.0, date=2018-10-20]
\changed
	\item Class renamed to from \ltxclass{problemset} to \bps.
\added
	\item A license header.
	\item |ganttschedule| environment.
	\item Additional keywords for |pseudocode| environment:
		\texttt{and}, \texttt{or}, \texttt{nil}, and \texttt{len}.
	\item |\ac| command for acronyms.
	\item An example document.
\end{version}

\shortversion{v=0.1.0, date=2018-10-19,
	changes=Initial beta as \ltxclass{problemset}.}
\end{changelog}
\end{document}
